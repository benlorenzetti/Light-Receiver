\documentclass{article}
\usepackage{graphicx}
\usepackage[colorlinks=false]{hyperref}

\begin{document}

\title{Hacking 10BASE--T Ethernet\\for Underwater Optical Communication}
\author{UC the Fish}

\maketitle

\section{Introduction}

Ethernet was originally developed for connecting multiple computers together
via a single, shared coaxial cable.
Time sharing the medium is managed with Carrier Sense Multiple Access with
Collision Detection (CSMA/CD). Each station monitors the cable
and waits until the line is quiet to begin a transmission (CSMA).
If two stations begin transmitting simultaneously, this is detected
(CD) and both stations stop transmitting and wait a random time
before retry.

\begin{center}
	\includegraphics[width=0.5\textwidth]{coax-topology.pdf}
\end{center}

Ethernet is an attractive protocol for underwater light communicion
because only blue and green light can propagate reasonable distances
in the ocean.
This makes water very much like a single medium, or possibly a
two channel medium, which all stations must share.

Furthermore ethernet handles noise and interuption gracefully,
with cyclic reduncancy checking built into the data packets
and retransmission attempts baked into hardware state machines.
The ocean is not a controlled environment, and other protocols
such as USB need to renegotiate the entire link if communication
from the master station is disrupted or altered.

UC the Fish plans to ``hack'' ethernet, in the positive sense where hack
means to repurpose for a new, exciting, but unexpected use.
With any communication protocol, the standard should be followed as closely
as possible so machines built by independent companies can coexist.
The ethernet standard has many substandards for different media,
including coax cable, twisted pair cable, and fiber optics.
None of the defined media have propagation properties similar to open water,
so for now a non--standard ``hack'' is needed.
But who knows, perhaps one day light+water will be a part of the standard.

\section{Getting IEEE Std 802.3}
IEEE Std 802.3 is the standard for Ethernet communication.
It is freely available online but is so large it is split
into multiple sections.
The first section (only 555 pages!) is available at
\url{https://standards.ieee.org/getieee802/download/802.3-2012\_section1.pdf}.


\section{10BASE--T Ethernet}

Ethernet is the most widely used local area network protocol for transmitting
information between PCs, servers, telephones, sensors, PLCs, and many other machines.
It defines physical form for communication signals,
topology and materials used to connect devices,
and how addressing and routing is done on such a network topology.
10BASE--T Ethernet uses a differential electric signal, on a twisted pair
of copper wires, with data clustered in ``frames'' that contain unique
addresses.

To expand on the above sentance:
\begin{enumerate}
\item The electric signal is differential to
	\begin{enumerate}
		\item Reduce common--mode noise
		\item Provide transformer isolation between PCs with different earth potential
		\item (HIGH and LOW logic levels are filtered out, so manchester encoding is used instead).
	\end{enumerate}
\item The electric signal operates at 10 MHz, with ???? voltage.
\item The length of twister pair cable is limited by speed of light.
	\begin{enumerate}
		\item Time required for a signal to propagate (round trip) between two furthest stations is ``slot time''.
		\item Maximum slot time is $51.2\,\mu s$.
	\end{enumerate}
\item Two fundamentally different network topologies are allowed.
	\begin{enumerate}
		\item Half--duplex mode (CSMA/CD) allows many stations to time share a single cable.
		\item Full--duplex topology is only two stations are connected to a Tx--Rx cable,
			but stations (routers) may have many connections to form star network topology.
	\end{enumerate}
\item Each station has a unique, 48 bit address for routing data ``packets''. Packets may be routed to
	\begin{enumerate}
		\item Individual MAC address
		\item A multicast to a group of address determined by software higher up in the computer
		\item Broadcast to all devices on the immediate topology cable.
	\end{enumerate}
\end{enumerate}




\end{document}
