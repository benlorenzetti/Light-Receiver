\documentclass{article}
\usepackage{graphicx}
\usepackage[colorlinks=false]{hyperref}

\begin{document}

\title{Hacking 10BASE--T Ethernet\\for Underwater Optical Communication}
\author{UC the Fish}

\maketitle

\section{Introduction}

Ethernet is the most widely used local area network protocol for transmitting
information between PCs, servers, telephones, sensors, PLCs, and many other machines.
It defines physical form for communication signals,
topology and materials used to connect devices,
and how addressing and routing is done on such a network topology.
10BASE--T Ethernet uses a differential electric signal, on a twisted pair
of copper wires, with data clustered in ``frames'' that contain unique
addresses.

To expand on the above sentance:
\begin{enumerate}
\item The electric signal is differential to
	\begin{enumerate}
		\item Reduce common--mode noise
		\item Provide transformer isolation between PCs with different earth potential
		\item (HIGH and LOW logic levels are filtered out, so manchester encoding is used instead).
	\end{enumerate}
\item The electric signal operates at 10 MHz, with ???? voltage.
\item The length of twister pair cable is limited by speed of light.
	\begin{enumerate}
		\item Time required for a signal to propagate (round trip) between two furthest stations is ``slot time''.
		\item Maximum slot time is $51.2\,\mu s$.
	\end{enumerate}
\item Two fundamentally different network topologies are allowed.
	\begin{enumerate}
		\item Half--duplex mode (CSMA/CD) allows many stations to time share a single cable.
		\item Full--duplex topology is only two stations are connected to a Tx--Rx cable,
			but stations (routers) may have many connections to form star network topology.
	\end{enumerate}
\item Each station has a unique, 48 bit address for routing data ``packets''. Packets may be routed to
	\begin{enumerate}
		\item Individual MAC address
		\item A multicast to a group of address determined by software higher up in the computer
		\item Broadcast to all devices on the immediate topology cable.
	\end{enumerate}
\end{enumerate}




\section{Getting IEEE Std 802.3}
IEEE Std 802.3 is the standard for Ethernet, a collection of communication
protocols for local area networks over a shared
physical media or a star topology of many private, point--to--point connections.
The 802.3 standard is freely available online but is so large it is split
into multiple sections. The first section (only 555 pages!) is available
at
\url{https://standards.ieee.org/getieee802/download/802.3-2012\_section1.pdf}.



\end{document}
